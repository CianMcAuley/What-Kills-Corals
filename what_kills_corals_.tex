\documentclass[11pt,a4paper]{article}
\usepackage[utf8]{inputenc}
\usepackage{amsmath}
\usepackage{amsfonts}
\usepackage{amssymb}
\usepackage[left=2cm,right=2cm,top=2cm,bottom=2cm]{geometry}
\author{Cian McAuley}
\title{What Kills Corals?}


%graphics with figure environment
\usepackage{graphicx}
\usepackage{hyperref}
\usepackage{breakcites}
\usepackage{caption}
\usepackage{subcaption}
\usepackage{natbib}
\usepackage{scrlayer-scrpage}
\linespread{1.7}

\hypersetup{colorlinks=true,citecolor=black,filecolor=black,linkcolor=black,urlcolor=black}


\begin{document}
\begin{center}

\title{
\huge Biogenic reef death and adaptation through time. 
}
\author{\Large \href{mailto:r02cm21@abdn.ac.uk}{Cian McAuley} 
\\University of Aberdeen
}
\date{\today}

\maketitle


\end{center}
\begin{abstract}

\end{abstract}

\section{Introduction}

\section{Factors affecting coral health}

%INCLUDE SUMMARY DIAGRAM OF FACTORS IF FEASIBLE

The following is a (non-exhaustive) set of factors that affect coral health and/or increase mortality. These parameters are assumed to be present and varied throughout Earth history; that is, even if the cause of a factor change is a new development previously unseen on Earth (e.g. land plant evolution, human agriculture), the change in the factor itself (Changes in atmospheric CO$_{2}$, terrestrial run off, global temperature) is not a new feature for shallow marine reef ecosystems or the Earth as a whole.

\subsection{Temperature}
Temperature is perhaps the most important factor for corals in the modern world, being the primary cause of bleaching events. While corals generally require warm waters, too high temperatures can cause an expulsion of the zooxanthellae, more commonly known as "bleaching".


\subsection{Terrestrial Runoff}

\cite{Fabricius2005} gives a detailed review and synthesis of the effects of terrestrial run off on coral reefs, focusing on the effects of dissolved inorganic nutrients (DINu), particulate organic matter (POM), light reduction and increased sedimentation, all of which increase with terrestrial run off. With the exception of POM, increases in these parameters have detrimental effects on corals and the reefs they formed, decreasing coral diversity (light reduction, sedimentation), calcification (DINu), changing morphology (sedimentation) while increasing growth for some species (POM). \cite{Fabricius2005} found that even short intervals of increased runoff can lead to long term effects due to the much higher sensitivity of coral embryos, polyps and juveniles to these parameters compared to their adult forms. 

\cite{Wooldridge2009} notes that although nutrient pulses from terrestrial run off are generally short (days to weeks), nutrients can be recycled around the food web of a shallow marine community such that the effects last significantly longer.

%This section probably needs expanding a BIT (emphasis on bit) more, and to explain how thoughts have changed wrt sedimentation and coral diversity, if the corals at the mouth of the amazon are actually doing decently. 

Nutrient enrichment from runoff can increase both the severity of coral diseases \citep{Bruno2003} and their susceptibility to bleaching \citep{Wiedenmann2013}. Increased dissolved inorganic nitrogen (DIN) can reduce the temperature threshold of coral bleaching \citep{Wooldridge2009}, which may be due to phosphate starvation of the symbiotic algae due to the imbalanced DIN supply \citep{Wiedenmann2013}, reducing the thermal and light thresholds for coral bleaching. \cite{Fabricius2005} noted (Found from other studies? What are the right terms) that increases in DIN lead to increases in zooxanthellae density. \cite{Wiedemann2013} hypothesises that this increase in zooxanthellae density would result in phosphate starvation where phosphate availability is normally limited, stating "[T]he most severe impact on coral health might actually not arise from the over-enrichment with one group of nutrients (for example, DIN) but from the resulting relative depletion of other types (for example, phosphate) that is caused by the increased demand of proliferating zooxanthellae populations".

\cite{Thompson2014} was able to conduct a "natural experiment" on the effects of runoff on a GBR coral community, finding that seasonal variability of suspended solids and chlorophyll-\textit{a} was enhanced during years with high discharge. Coral cover was stable for large/adult corals, but there was a clear shift in abundance and composition of juvenile corals across the transition from below-median-rainfall conditions to above-median-rainfall conditions. Coral disease was positively correlated with the proportion of fine grained sediment, and also most prominent during this transition, indicating that the "first flush" of years of sediment accumulation within the river during the dry years had the negative effect on the corals. They also suggest that this first exposure may have removed the most susceptible colonies, leaving behind only those that were less susceptible.

\cite{Uthicke2010} is a long and relevant paper. To do soon.

as is \cite{Erftemeijer2012}


\subsection{Competition}

%NEED NEW AND BETTER SOURCES

\cite{Fabricius2005} also reviewed the effects of DINu, POM, light and sedimentation on other marine organisms that interact with corals, noting that fertiliser application (and therefore potentially eutrophication in general) led to a 10-fold increase in erosion by algal microboring, leading to weaker substrate and leaving corals vulnerable to storms. Macroalgae directly compete with corals in well-lit environments, where they can overgrow and damage corals, and even produce anoxic zones when they collapse. Fabricius reports that algae are limited by nutrient availability, and therefore increases in DINu and POM can lead to their growth with a negative impact on the reef.
In areas of low light and/or organic enrichment, \cite{Fabricius2005} notes that filter feeders such as bivalves outcompete corals for space where in normal reefs they would not. As they tend to fill separate niches, declining coral cover and increased heterotroph filter feeding are generally independent with respect to increasing runoff.
Fabricius reports that infections occur and spread quicker when inorganic nutrients are raised, though disease prevalence has also been linked to higher sea water temperatures.
The Crown-of-thorns starfish (\textit{Acanthaster planci}) is a predator of corals, and outbreaks of them can cause great damage to coral reef communities. Fabricius notes from experimental studies that the larvae of this starfish better survive with high phytoplankton concentrations, which increase with nutrient content, and that outbreaks can even form at significant distance from runoff sources due to larvae from previous outbreaks being carried by currents.

\subsection{Dissolved Inorganic Nutrients}

Increases in DINu can increase zooxanthellae density (Snidvongs and Kinzie, 1994) but decrease calcification by up to 50\% (Marubini, 1996), and also decrease primary productivity (Nordemar at al., 2003). \cite{Wiedemann2013} hypothesises that this increase in zooxanthellae density under increased DIN would result in phosphate starvation when phosphate availability is normally limited, 
 Short term exposure to high dissolved inorganic nutrients does not kill or harm colonies, but longer term exposure can change reef metabolism. However, many studies that Fabricius reviewed were conducted at DINu levels unrealistic for natural environments: Koop et al. (2001) studied the effect of high daily nutrient fluxes that would not be expected in nature, and DINu (both nitrogen and phosphorous compounds) are quickly taken up by phytoplankton in the water column and therefore rarely stay elevated for long periods of time. Severe effects of dissolved nutrients are therefore limited to restricted environments such as lagoons or upwelling areas with a steady supply of nutrients, and reef formation is lower in regions of upwelling as a result (Birkeland, 1987), though cooler temperatures may be the more dominant effect.

DINu have a large effect on coral reproduction, negatively affecting egg sizes, fertilisation rates, and embryo development when elevated (Ward and Harrison, 2000), and at much lower limits than adult corals could handle.


\subsection{Light availability}

The light reduction caused by particle shading in turbid waters is minimal in shallow warer, but increases with depth. \cite{Fabricius2005} reports that reduced light and sedimentation can cause lower calcification, thinner tissues, and increased mortality (Walker and Ormond, 1982; Telesnicki and Goldberg, 1995) and that turbidity can cause changes in community structure, compressing depth zonation and reducing species richness (Loya 1976; Acevedo and Morelock 1988, etc on page 5 of the paper). The size of particles also has a significant impact with finer particles such as clays being resuspended easier and taking longer to settle than coarser particles (Te, 1997). 

Light requirements vary between species. Few can survive in deeper waters or high turbidity environments, leading to low coral diversity in these places. Surprisingly however, the "perfect conditions" also often have low diversity: In shallow waters with zero turbidity, phototrophic species with rapid growth outcompete heterotrophs and slower growing species, pushing them out of the environment. Cornell and Karlson (2000) report that diversity is therefore often greatest at intermediate light levels. 

Light levels have an effect on both coral reproduction and recruitment, with fecundity decreasing in low light conditions (Fabricius 2004). Coral larvae also use light quantity to determine their settling site, leading to them preferentially colonising upper surfaces in low light levels, where sedimentation is higher (Birkeland et al., 1981). 

\cite{Morgan2020} Light limitation and depth variable sedimnetation drives vertical compression on turbid coral reefs.

\subsubsection{Coral Photosynthesis}

Light availability of course only matters if an organism is gaining something from it. For modern corals, it is of course important due to many of them getting energy through photosynthesis with their dinoflagellate symbionts (\textit{Symbiodinium}), and this symbiosis can be traced back to around the end Cretaceous extinction \citep{Lipps2016photosynthesis}. However, the occurence of photosynthesis in older sclerectian corals and especially the completely extinct rugose and tabulates of the Palaeozoic is much less known. 

\subsection{Particulate Organic Matter}

Increasing POM is beneficial for some, but not all coral species (mostly heterotrophic ones), potentially providing growth benefits such as increasing linear extension (Citing Meyer and Schulz 1985) and increased tissue thickness (Anthony and Fabricius, 2000), while decreasing skeletal density (Lough and Barnes, 1992). Above certain levels, POM effectively becomes a simple sedimentary particle as feeding becomes saturated, though this level is dependent upon coral species. Increased POM leads to growth increases of both the coral and zooxanthellae, while an increase in DINu is preferentially taken up by the zooxanthellae (Dubinsky and Jokiel, 1994).

\cite{Fabricius2005} notes a modal change in coral growth and calcification along eutrophication gradients: "In areas of intermediate turbidity where particulate and dissolved nutrient loads were high, corals had higher concentration of photosynthetic pignments, calcification, gross photosynthesis and respiration compared to a cleaner site (Tomascik and Sander, 1985; Marubini, 1996)." 

POM can negatively affect fertilisatrion rates, larval development/survival, and settlement (Gilmour 1999), and Fabricius notes that it is unknown to what extent young corals feed on POM compared to the adult colonies.


\subsection{Direct sedimentation effects.}

While sedimentation flux can have a large control on the above three parameters, it also plays a direct role in coral health.
Sedimentation has a progressively negative effect on corals, with low sedimentation increasing respiration and reducing net photosynthesis (Abdel-Salam et al. 1988) and high sedimentation reducing coral cover, species diversity (only the most tolerant can survive), net productivity, rates of reef accretion and changing morphology to favour branching forms that can more easily "shake off" sediment (citing Rogers 1990) and reducing linear extension (Cotes and Risk 1985; Dodge et al 1974). Removing sediment from coral branches also increases metabolic costs to the colony (Telesnicki and Goldberg (1995).
Wesseling et al. (2001) found that sedimentation fluxes can lead to larger colony sizes, though this is primarily due to sediment reducing coral recruitment rather than any benefits the nutrients in the sediment might bring. 
High sedimentation rates can kill coral tissue within a few days, but similar effects can be achieved with lower loads over a longer duration. Damage to coral also depends on sediment type, with tissue damage increasing with increased organic content and bacterial activity, and decreasing grain sizes causing more abrasion (Hodgson, 1990b; Weber et al., 2004). 
Toleration of sedimentation varies quite widely between coral species, with those forming large, branching colonies performing better than smaller colonies with thinner tissues and flat surfaces that sediment may settle on (Rogers, 1990). This generally leads to a reduction in coral diversity in response to sedimentation.

%Acevedo and Morelock (1988) found that coral cover and diversity increased with distance from the sediment source. 
%CHECK ALL THESE PAPERS FOR THE ACTUAL INFO IN THE ABSTRACT OR DISCUSSION AT LEAST

Coral reproduction and recruitment are much more sensitive to the above effects than adult corals, with Fabricius reporting that mortality thresholds for sedimentation are an order of magnitude lower for coral recruits compared to already established colonies. Polyps that do survive are often those that attached to downward facing surfaces that are then more limited by light leading to slow growth. Short term exposure to elevated sedimentation rates can have longer term knock on effects for coral populations, as younger corals or unestablished polyps are more easily wiped out, thereby slowing the recovery of the reef.

Coral reefs in poor quality waters have slower recovery times from disturbances and are more susceptible to disease and predatorial outbreaks \citep{MacNeil2019}, though they found that exposure to river-influenced water plumes lead to reefs in these waters are generally more resistant to bleaching, as higher turbidity waters reduce the exposure to light stress and more generally pushes coral communities toward species that are more tolerant to disturbance.  




\subsection{Effects from other organisms}

Algae (both macro and micro)

Phytoplankton

echinoderms

"Marine snow", formed of polysaccharides exuded by diatoms and bacteria, can kill newly settled coral recruits (Fabricius et al., 2003) where low sedimentation rates alone could not. 

In the modern day, macroalgal blooms on coral reefs are often thought to be linked to declines in algal-herbivory, allowing algae to colonise the substrate and overwhelm the ability of remaining grazers \citep{Williams2001,Hughes1999} (CHECK HUGHES NOW (NAH MAYBE LATER)), and that therefore coral cover would be inversely related to algal cover and low coral cover would lead to algal dominance. However, \cite{Bruno2007} found that macroalgal cover has not significantly increased in most reefs that have low coral cover (two thirds of reefs with under 10\% coral cover had under 20\% algal cover), suggesting that other organisms such as sponges were the primary beneficiaries from coral cover loss.

Corals are also under threat from sponges \citep{Elliott2016} such as \textit{Terpios hoshinota}, which overgrows corals and kills them. The sponge has been found across the Pacific ocean and appears to be expanding westward into the Indian ocean. \cite{Elliott2016} found that some colonies of \textit{Montipora aequituberculata} were able to out-compete sponges that were overgrowing them.

\subsection{Adaptations}

%Corals can adjust to lower light levels within 5-10 days by increasing in size and amount of chloroplasts, but since light exposure on inshore reefs goes through a fivefold range just due to tides, clouds and resuspension, photoacclimation does not significantly enhance gross productivity (Anthony and Hoegh-Guldberg, 2003).

Corals have several mechanisms to reduce the effects of sedimentation, such as changing growth morphology from branching or flat forms to dome shaped one, and using mucus and cilia to physically move sediment off their surfaces \citep{Logan1988, Stafford1992}. The place of settlement and growth direction is a large factor in a corals efficiency to remove sediment from itself, with more steeply inclined top surfaces removing sediment faster than those that were horizontal \citep{Logan1988}. Differences in morphology can control pH at the surface of the Coral \citep{Chan2016}.



\cite{Todd2008} notes that morphological plasticity, that is the ability of corals of the same species to have different growth forms dependent on their environment, arises in two main ways:
\begin{itemize}
\item Different morphologies are due to different genotypes (i.e. morphology is set in "at birth"), and therefore corals with the best suited genotype for an environment will survive \citep{Ayre1988}
\item  Corals are phenotypically plastic and and change their morphology to respond to their environment as they grow. \citep{Muko2000}
\end{itemize}

\cite{Muko2000} found that transplanting coral fragments from plate colonies of \textit{Porites sillimaniani} into differing light conditions caused the corals to differ morphologically, with those in high light conditions developing coral branches while those in low light conditions remained flat. \cite{Todd2004} found a similar relationship between light and growing coral morphology in transplanted fragments of \textit{F. speciosa} \textit{D. heliopora}, but did not find relationships between coral morphology and sediment regime/load, or morphology and water energy. However, it is known that there is preference for dome-like morphology under increased sedimentation \citep{Stafford1992}, so it may be possible that species adaptations to sedimentation and water energy are due to genotypic survival patterns on a community rather than the "true plastic" responses such corals have to light within a single colony's lifetime. \cite{Todd2008} was wary to ascribe morphological plasticity to corals in general, however, as at the time only 20 species had been tested for it. Coral Morphology can also be driven by acidification \citep{Tambutte2015,Alison2022}, albeit in changes to the skeletal structure (higher porosity) rather than to the overall shape of the colony.

\cite{Million} found that there is genetic variation in plasticity among \textit{Acropora cervicornis}, and that plasticity was positively correlated with growth rate and survival. They predict that future generations will become more plastically adaptable due to positive selection for those traits. 

the \textit{M. aequituberculata} corals mentioned above \citep{Elliot2016} were able to out compete encrusting sponges through morphological plasticity. While normally a foliose coral, it was able to overgrow it's competitor when it was in an encrusting form, growing over already established colonies of \textit{T. hoshinota} rather than directly attacking the growth front of the sponge. \cite{Elliot2016} concludes that if encrusting morphology of corals become more common in the future (presumably as a form more resistant to competition stress), it would lead to a "reduction in the 3-dimensional structure of these reefs".

\cite{Hennige2008} physiological adaptation

\cite{Safaie2018} teperature variation leading to bleaching resistance.



Some coral species can even digest sediment for nutritional benefit \citep{Rosenfeld1999}. \cite{Camp2017} describes how corals  adapted to a harsh lagoonal environment through "heterotrophic plasticity", turning to food sources in the water column. 


\section{The Past}

%A diagram, potentially even full page, on the types of reefs that existed through time would be useful

While the famous adage "The present is the key to the past" \citep{Lyell_1837} does not always hold true with regard to both Earth and the life on it (For example, \cite{McMahon2018b} notes that we cannot observe "pre-vegetation rivers" in the modern day), comparisons can be drawn and inferences can be made on the nature of organisms or geological processes in deep time if they are similar enough to those in the present. Although the rugose and tabulate corals of the Palaeozoic are of a completely different order than the sclerectinians of the Mesozoic to present, they share common modes of life (sessile autotrophs and heterotrophs), habitat (predominantly tropical shallow marine) and physical features (calcified skeleton, symbiotic zooxanthellae). It is therefore probable that the stresses that negatively affect sclerectinian corals in the present would have also affected the rugose and tabulate corals that came before them.  It is also possible that they would have developed similar mechanisms or behaviours to deal with such stressors, such as growing with an inclined calice in response to sediment stress \citep{Logan1988}. Though present day corals have to deal with the unprecedented scale of human civilisation, the actual nature of these impacts (oceanic warming and associated acidification, enhanced terrestrial sediment flux, pollutants, eutrophication and oceanic anoxia, etc.) are known to the geological record and have caused reef ecosystem collapse several times \citep{Lipps2016photosynthesis}. It is arguable that Palaeozoic corals and other reef organisms passed through greater relative changes from the prior norm, such as the Frasnian-Fammenian extinction that completely wiped out the Stromatoporoid-Coral reef systems of the time, from which they never truly recovered before their complete extinction in the End Permian Extinction, something that has not (yet) occurred for sclerectinian corals in the present. 


%various disparate sentences that dont really fit together yet
\subsection{Expansion and Collapse}

\cite{Copper1994} gives an overview of reef expansions and collapses through geological time, giving particular emphasis to the influence of the oscillating climate cycle. Reefs greatly profited during times of global warming, producing carbonate platforms exceeding 5 million km$^{2}$, while during collapses such as the Frasnian-Fammenian, the total extent of reefs was at best 1000 km$^{2}$. Several reef tracts in the Silurian were longer than the Great Barrier Reef of today. \cite{Lipps2016} gives a review into what organisms were dominating reef environments and what environmental changes/events they were responding to, noting that dominant biota have been eliminated by events that each correspond to periods of rising green house gases and ocean acidification.

\cite{Lipps2016} propose that reefs, despite how we think of them today, are not particularly fragile ecosystems, as many individual ancient reefs existed for millions of years. However, there is a difference in resolution between the geological past and present that is difficult to reconcile: We cannot observe current reef systems for thousands or millions of years to investigate whether they are resilient enough to recover from anthropogenic climate change, nor can we investigate a particular hundred year long interval in the Palaeozoic and examine changes in reef health during that time. While the factors and mechanisms that affect reef health are likely similar throughout geological time, our understanding of the effect of duration of a change or event on reef systems is limited by our perspective resolution on both the past and present.

\cite{Leinfelder1999} give a review into the evolution of the "building blocks" of reefs from the Precambrian to the present day. They posit that while reefs got more complex through time and evolved to better manage energy, they also became less flexible with regard to change. They found that the "reef window", the region in which reefs could live, widened until the Cenozoic, and that it has since narrowed due increasing modular complexity and interdependence between organisms. Two windows now exist: the traditional shallow marine region (though now narrower than its height in the Jurrasic-Cretaceous) and a deep water window where heterotrophic `cold water` corals and other organisms can live. %Joyce says give examples here. I guess there are corals/deep reefs round the UK/continental shelf? Or mention GARS here for the first time?

\cite{Leinfelder1999} found that the reef window was wide from during the mid-late Devonian (until the Frasnian-Fammenian), with reefs existing down the continental shelf. Then, until the Carboniferous, corals were mostly restricted to deeper environments, but although building blocks appeared at much reduced abundance, the reef window remained quite wide. The Jurassic reef window was large, with reefs appearing in stressed environments, though diversity at these edges of the reef window was low - reef systems as a whole were flexible, but reefs in any given niche, especially those at the edges, were not. Rudist bivalves were a key component in many reefs in the Cretaceous, and may have been more dominant in stressed environments \citep{Leinfelder1999}, or may have been dominant across ecosystems due to ocean chemistry at the time \citep{Ries2006}. At the same time, coral dominated reefs began to switch to the more modern coral-coralline algae type reefs, which are more sensitive to environmental change and likely lead to a narrowing of the coral reef window. Modern reef windows are narrower because corals had difficulty colonising the marginal settings that the extinction of the rudists left behind.

%maybe make some updated versions of the plots that leinfelder made for fig 3
\subsubsection{Calcitic and Aragonitic seas}

One long term secular change that could control reef composition and expansion is the 

\cite{Ries2006} gives an overview on how Sclerectinians performed in the Mesozoic due to the changing seawater chemistry: They began to flourish and became reef builders in the Late Triassic through to the Early Cretaceous due to conditions favouring precipitation of aragonite (High seawater \textit{m}Mg/Ca), but rudist bivalves became the primary reef and carbonate producers in the mid-Cretaceous due to calcite becoming the favoured precipitate. From the beginning of the Oligocene, where the \textit{m}Mg/Ca ratio rose above 2 again, sclerectinians returned to being the major reef builders that they still are today. \cite{Ries2006} found that some sclerectinians can produce calcite under low \textit{m}Mg/Ca conditions, but that they grow slowly, explaining why they lost dominance in the Cretaceous but did not disappear entirely.
For reference, seawater chemistry favoured calcite deposition for nearly all of the Palaeozoic, only favouring Aragonite deposition in the Carboniferous. This may explain why rugose and tabulate corals had difficulty re-establishing reefs after the Frasnian-Famennian extinction.



\subsection{Cambrian Reefs}
Metazoan reefs in the Cambrian were primarily archaeocyath bioherms, with tabulate corals forming a minor part but never being structurally relevant \citep{Copper1994}, with cyanobacteria being  binding agent \citep{Lipps2016}. 

\subsection{The Frasnian-Famennian (Late Devonian) extinction}
The end-Frasnian extinction effectively wiped out the stromatoporoid-coral reef ecosystem that had been proliferating since the Silurian. Reefs as a whole were not removed, but it would take several million years before corals would play a significant part again. There are several proposals for the root cause of the extinction, but most generally agree that the proximal cause was ocean anoxia events.

\cite{WuFengGong2013} proposed a kill mechanism for this event observed in Devonian stromatoporoid-coral reefs in South China. They found corals and stromatoporoids that had been smothered by algae and bacteria (\textit{girvanella, rothpletzella}), and skeletal damage holding soft tissue suggests that bacteria and algae even bored into reef building organisms to kill them directly. Bacterial-algal blooms are well documented for the Devonian of South China, and similar sediments indicating anoxic events at the time of the Frasnian-Fammenian (F-F) extinction (such as the Lower and Upper  Kellwasser layers) have been found around the world \citep{Bond2005, House1985, Joachimski2001, Wendt1991}. %CITE CITE CITE
\cite{WuFengGong2013} also suggested that before this interval, a general balance between reef builders and the invasive bacteria and algae existed, where corals and stromatoporoids were able to resist invasion and effectively self heal. Under the deteriorated conditions of the Late Devonian seas, however, bacteria and algae flourished while the defensive ability of corals and stromatoporoids declined, leading to the deaths of the reefs. \cite{WuFengGong2013} also found that stromatoporoid-coral reefs were also severely impacted by Bacteria-Algae blooms in the Givetian-Frasnian, and not just the Frasnian-Famennian.



\subsection{The Invasion of the Land}

One intriguing possible cause for the F-F extinction and collapse of coral reefs is the "Land Plant Weathering Hypothesis", first put forward by \cite{Algeo1995} and expanded in \cite{Algeo1998} and \cite{Algeo2010}. In essence:

\begin{itemize}
  \item The evolution and more importantly increasing dominance of arboresence and deep (1 m+) roots in vascular land plants in the Late Devonian lead to greater rock and soil weathering. 
  \item The evolution of seed habits lead to greater colonisation of upland areas, leading to a larger proportion of weathered land.
  \item In the short term, these increases in weathering lead to increased sediment yield and nutrient fluxes to the oceans. 
  \item These nutrient pulses would cause eutrophication leading to algal blooms (as per \cite{WuFengGong2013}) and eventually ocean anoxia.
  \item In the longer term, land plants stabilise the landscape and sediment flux shifts from weathering-limited to transport-limited (weathering products get stuck in rivers and other pre-ocean sinks), reducing the sediment and nutrient yield.
\end{itemize}

%It's a neat theory, and one that I agree with because it's essentially the basis of my PhD title.

\cite{Algeo1995} argue that the Late Devonian extinction events are temporally related to palaeobotanic developments, with ocean anoxia coinciding with rapid increases in the maximum size of vascular land plants, and the Kellwasser event in the F-F extinction occurs during an interval of Archaeopterid dominance. Recent timelines push the evolution of arborescence and roots to the Mid Devonian or even early Devonian \citep{Hetherington2018}, well before the F-F boundary, but \cite{Algeo1995} note that first appearance of such a trait is much less important than it's rise to dominance. 
\cite{Davies2010} suggests that the first appearance of plant fossils in the sedimentary record would generally indicates that they were sufficiently abundant to interact with sedimentary systems, but does note that there a lag time between first appearance and effect on the terrestrial environment could mean that a "critical mass" needed to be reached.

\cite{Algeo1998} place this increase in influence of land plants on the ocean in the Mid Devonian with the evolution of deeper roots, arguing that earlier bryophyte grade plants had negligible impact on their physical environment and weathering due to their small size, restriction to wet lowlands, and shallow or non-existent roots.
 However, there's evidence that early lad plants significantly interact with their evironment even without roots:
\cite{Schumm1968} noted that early land plants could have had a significant impact on alluvial environments. 
\cite{Davies2010} supports this view, suggesting that from the Middle Ordovician onward (first evidence of cryptospores), early vegetation would have increased chemical weathering, production of fine sediment, and alluvial storage. They propose that prior to the evolution of tracheophytes, "the seaward transportation of all sediment was increased".
 Land plants don't just increase weathering, they also increase riverbank stabilisation through their roots \citep{McMahon2018b} and sediment retention through both their roots and above surface expression. 
\cite{McMahon2018a} show that land plants had an impact on the planet from the Silurian onward, increasing the amount of mud grade sediment retained on land in the form of an irreversible increase in alluvial mudrock, predating the evolution of larger, deeper rooting land plants. 
\cite{Zeichner2021} supports this view, finding that clay size particles flocculate much more readily with the presence of plant organics in the water than without them, greatly increasing mud deposition. 
Land plants in the pre-Devonian may have been small and primarily restricted to lowland alluvial habitats such as braided river channels \citep{Davies2010}, but these are the environments that most sediment headed for the ocean would have passed through. Plants of bryophyte grade still have an above surface expression to trap or baffle this sediment, and plant organics suspended in rivers would allow clay and mud size sediment to flocculate. 
\cite{Davies2010} notes that "even dense strands of \textit{Cooksonia} would have been vulnerable to erosion and destruction during flood events", but that following the evolution of these early tracheophytes, there was still increased alluvial storage of fine grains, which would later be supported by \cite{McMahon2018a}. 
  \cite{McMahon2018b} contend that such behaviour began almost as soon as land plants evolved in the Ordivician, leading to increases in bank stability and the rise in frequency of meandering river channels compared to shallow and braided rivers in the rock record. 
 \cite{Zeichner2021} predicts that plant-induced flocculation would lead to muddier and more cohesive channel banks, restricting lateral migration of channels and braiding. The smaller, more sinuous profile and less migratory nature of post-vegetation rivers could lead to less sediment being remobilised on flood plains, or at least not as regularly. %Is this true? Feels intuitive but how would you go about getting evidence for it?




However, even though the proportion of mudrock on land increased, this does not necessarily mean that the influx of sediment to the ocean was reduced, or could not later increase: If Silurian bryophytes caused a 5 fold increase in sediment retention, a 20 fold increase in weathering in the Devonian due to the evolution of deeper roots could still produce an increase in flux to the oceans. The only true record of sediment yield to the oceans is in ocean sediments, but large parts of this record - especially for the mud fraction that settles off the continental shelf - has been erased by tectonic subduction. It may even be that the middle Devonian evolution of deep roots did not increase physical weathering due to the stabilising impact of roots, but that chemical weathering and dissolved nutrient flux to the ocean still increased due to increased infiltration of rocks and soils. \cite{Davies2010} contend that the increased storage capacity of post-deep-rooting soils reduced surface run off, but it is unclear if this is talking about an absolute change or a relative one. The proportion of retained runoff may have increased from 1\% to 20\%, but the absolute volume of sediment run off reaching the ocean could still have increased dramatically due to the increase in weathering that those roots caused.
 
 %Need a database study on the proportion of mudrock in shallow marine sediments at least.
Geochemical isotopes, namely $^{87}$Sr/$^{86}$Sr and $^{187}$Os/$^{188}$Os \citep{Percival2019} have been used as marine proxies for continental weathering.

It is therefore proposed that land plants may have had a positive effect on shallow marine ecosystems before the Devonian, where small plants were able to trap and baffle sediment in the predominantly shallow rivers without greatly increasing weathering and sediment production due to their shallow/nonexistent roots, while the presence of plant biomaterial would lead to greater mud flocculation and deposition. These factors would lead to a lower nutrient flux to the ocean which allowed corals and reefs to thrive in the Silurian. In the Mid-Late Devonian, the increases in weathering from the evolution of arborescence and deep roots became much larger than any gains in sediment retention through the effect roots had on stabilising the alluvial plain, leading to large nutrient pulses in line with the hypothesis of \cite{Algeo1995}, potentially becoming the root cause for the reef collapse at the Frasnian-Famennian Extinction.

Plants did not just affect the terrestrial environment directly, but also greatly affected the atmosphere and climate by drawing down CO$_{2}$ through photosynthesis. Such large decreases in pCO$_{2}$ would decrease the average global temperature significantly in the long term, however modelling by \cite{LeHir2011} predicted that the reduction in albedo and changes in soil properties caused by the colonisation of the land would counteract the decrease in CO$_{2}$, leaving the climate warm for much of the Devonian. These constant warmer temperatures would promote further silicate rock weathering and CO$_{2}$ consumption, which could have lead to a precariously balanced situation where global temperatures were significantly higher than the pCO$_{2}$ level could support, maintained only by the higher albedo granted by land plant cover \citep{LeHir2011}. A collapse in plant cover could then lead to a collapse in the warm climate, resulting in...

\subsection{The Permian-Triassic Extinction}

The End Permian was the single greatest extinction event in Earth's history, caused by . Both orders of stony corals, the \textit{Rugosa} and the \textit{Tabulata}, became extinct, along with over 90\% of all species. The extinction was protracted over millions of years, with over 70\% of coral families in China becoming extinct at the Guadalupian-Lopingian boundary, some 7 Myr before the end of the Permian, and the general pattern of extinction being massive forms first, followed by fasciculate corals, while small, simple solitary corals lasted until the latest Permian \citep{Wang2007}. 

\subsection{The Mesozoic and Cenozoic}

%Insert figure 1 from Ries and Stanley 2006 if allowed
Corals would not appear in the fossil record again until the Middle Triassic, tens of millions of years later. These new stony corals, \textit{Sclerectinia}, are not closely related to their ancient ancestors and predominately form their skeletons out of aragonite rather than the calcite preferred by the rugose and tabulates. 

\subsection{A paradox and general thoughts that will never make it into a finished product}

How is it that corals were able to exist and even thrive throughout the Phanerozoic, when CO$_{2}$ levels were sustained at over 1000 ppm and temperatures could be significantly higher than today? \cite{Pandolfi2011} explains that $\Omega$, a mineral's saturation state, is effectively decoupled from long term, steady-state conditions of high CO$_{2}$ and low ocean pH, because negative geochemical feedbacks increase alkalinity and calcium availability.

WARNING: UNFOUNDED SPECULATION AHEAD.

 \cite{Pandolfi2011} attribute this increase in calcium availability to the increased rock weathering due to high CO$_{2}$, but such increased weathering could also come from land plants, which could explain why corals were able to survive. If so though, why were they thriving before the land plants existed?



\section{The Present}

Corals in the present day are impacted not only by ongoing environmental forces that they have dealt with through geological time, but also human activity. Deforestation has lead to dramatic increases in terrestrial run off, while the use of fertilizers has greatly increased the nutrient content of that run off (Tilman et al. 2001; Smith et al. 1997, \textbf{read these sources from Fabricius 2005)}. 

Human development brought changes to the marine environment, primarily in the form of increased delivery of sediment, nutrients and pollutants. Despite the detrimental impacts that increases in sedimentation and nutrients can have, coral reefs can still be found living near or even in the waters of tropical coastal cities. \cite{Heery2018} investigated these "urban" coral reefs, those living in close proximity to major cities such as Singapore, Jakarta and Hong Kong. Singapore for example has "extremely compact" patch and fringing reefs, with water visibility as low as 2 m (down from 10 m in 1960). Corals are mostly restricted to 3-6 m depths, as macroalgae form a canopy above. Despite clearly being highly stress resistant, reef cover in Singapore has decreased from 32.2 km$^{2}$ in 1922 to 9.7 km$^{2}$ in 2011. In common between each of the reefs reviewed, \cite{Heery2018} found that reefs in these urban environments are dominated by stress resistant domed coral taxa such as \textit{Porites} and \textit{Montipora}, forming reefs of low to medium structural complexity, while branching taxa such as \textit{Acropora} were rare even as far as 20 km away from the city. Decreases in coral cover with progressive urbanisation over the 20th century is widely documented, and the bathymetric range of the coral reefs decreases with increasing turbidity. \cite{Heery2018} also documents the rapid recovery of urban corals from bleaching events: in the Singapore reef, the urban corals had exceeded the coral cover before the 1998 bleaching event in as little as five years, outperforming reefs in more ideal, "remote" environments. %they cite Guest et al. 2016.

Corals are known to be declining world wide, with cover across the Caribbean decreasing 80\% in the three decades leading up to the 2000s \citep{Gardner2003}.

%They found that reefs became bathymetrically compressed with increasing turbidity and the reefs were generally low diversity, dominated by domed growth forms.

\subsection{The Climate}
\label{Present Climate}

The latest IPCC report (\cite{IPCC2021WG1}, also referred to as IPCC2021 throughout) states that global mean Sea Surface Temperature (SST) has increased by 0.88\textdegree C since the beginning of the 20th century, with a large majority of this warming having occured since 1980. Since the 1950s, most warming has been concentrated on the Indian Ocean and Western boundary currents, while the Southern Ocean, North Atlantic and equatorial Pacific have slower warming or even minor cooling. Nevertheless, IPCC2021 predicts that 83\% of ocean surface waters will warm over the coming century in all predicted scenarios. Tropical SSTs are now warmer during "normal" La Ni\~{n}a conditions than they were during El Ni\~{n}o events thirty years ago \citep{Hughes2018a}. 
Marine heatwaves also became more common over the 20th century, with most heatwaves between 2006 and 2015 being directly attributed to anthropogenic warming (IPCC2021). 
The same is true for increasing ocean acidity, where there has been a long term increase in pH over the past 50 Ma, while pH as low as modern times is uncommon over the past 2 Ma and unprecedented over the past 26 Ka. pH in the ocean interior has been observed to decline in all ocean basins over the past 20-30 years. 
Oxygen levels have dropped in many regions compared to 20 years ago. High and low salinity environments have shifted closer to their own extremes respectively since 1950 \citep{IPCC2013TS}.

IPCC2021 also notes that the geographic range of many marine organisms has shifted poleward and toward greater depths. This can be observed in corals and the reefs they form (CAN IT? PROVIDE SOURCE, I'M SURE I READ SOMETHING ABOUT THIS).

\subsection{Bleaching}
\label{Present Bleaching}
Under stressed conditions, corals can expel their symbionts in a process known as coral bleaching. While this can be caused by a variety of factors such as sedimentation or reduced salinity \citep{Van1995}, the most extensive and severe events are caused by high sea water temperatures \citep{Baird2018}.  Long term bleaching leads to high coral mortality as the corals can no longer efficiently take up nutrition. 
Once a relatively rare event that reefs could recover from over time, the median time between severe bleaching events is now only six years \citep{Hughes2018a}. Regional scale bleaching was rare before 1980, global bleaching events started to occur from the 1980s onward in conjunction with stronger El Ni\~{n}o events due to climate change, and that in recent decades regional bleaching events are now occurring outside of El Ni\~{n}o conditions. \cite{Hughes2018a} states: "... The link between El Ni\~{n}o as the predominant trigger of mass bleaching is diminishing as global warming continues and as summer temperature thresholds for bleaching are increasingly exceeded throughout all ENSO phases".

Bleaching does not affect all corals equally. There appears to be strong selection for corals and their symbionts to evolve bleaching thresholds that are above but still near the highest temperatures expected in a given locality: Cooler locations grow corals with lower bleaching thresholds, while corals in the Arabian Gulf have thresholds 10\textdegree C higher than summer maximum temperatures compared to corals of the same species in cooler areas \citep{Hughes2003}. \cite{Hughes2003} investigated nine species of coral in French Polynesia, finding that bleaching impacted different coral species very differently: While 100\% of two \textit{Acropora} species became bleached, only around 30\% of \textit{Pocillopora verrucosa} bleached and 0\% of \textit{Porites lobata} or \textit{Leptastrea purpurea} bleached. \cite{Pandolfi2011} notes that projection models predict that these susceptible taxa such as \textit{Acropora} will reduce in abundance compared to massive or encrusting taxa with slower growth rates, and that observations following bleaching events confirm such community change. 

Worldwide bleaching events occured in 1998, 2010 and 2016.  \cite{Eakin2016} argues that the "2016" event started in 2014 even though El Ni\~{n}o conditions did not form until 2015-16, making it the longest coral bleaching event on record when it ended in 2017.  More than 70\% of corals worldwide experienced bleaching during this event, additionally making it the most widespread event on record. \cite{Hughes2018b} described it as "a watershed for the Great Barrier Reef", and that it had triggered a permanent transition to a degraded and less diverse but more heat tolerant assemblages on the Northern parts of the GBR, which until that point had been the most pristine region of the reef. 

There is strong evidence that these Marine Heatwaves (MHWs) are directly linked to coral bleaching \citep{Eakin2016,Hughes2018a,Hughes2018b}

Coral bleaching is exacerbated by poor water quality, and in particular anthropogenic surface run off. \cite{Wooldridge2009} found a link between dissolved inorganic nitrogen (DIN, predominately from fertilised crop lands) and a lowering of the upper thermal threshold for bleaching on inshore reefs of the Great Barrier Reef (GBR). DIN reductions of 70+\% could increase thermal bleaching thresholds by up to 2.5\textdegree C along the Australian coastline, indicating that while coral bleaching is a global problem, its impacts can at least be somewhat reduced by local land and reef management \cite{Wooldridge2009}.

\subsection{Erosion}

Similar to how land plants revolutionised terrestrial weathering in the Devonian,  human activities such as agriculture, mining and construction now play a greater role in geomorphic change than geological processes such as orogeny and uplift. Reviewing the literature, \cite{WilkinsonMcElroy2007} found that while the mean rate of erosion for the whole Phanerozoic was 5 Gt/Yr of sediment, Pliocene values were 16 Gt/Yr, and estimates for modern day sediment transport in rivers  range from 13.5-20 Gt/Yr (citing papers such as milliman and meade 1983, Berner and Berner 1987, Harrison 1994, etc.). 
 
\cite{WilkinsonMcElroy2007} also found that the modes of erosion for natural and anthropogenic sources were almost mutually exclusive: For much of Earth's History, rainfall and the rivers they flow into were the most important geomorphic processes, and they estimate that 83\% of global river sediment flux derived from the highest 10\% of Earth's surface where more precipitation falls and streams/rivers form. In contrast, Wilkinson and McElroy found that 83\% of cropland erosion occurs over the lowest 65\% of the surface. Not only that, but they found data suggesting that cropland denudation in the present day produces 75 Gt/Yr of sediment, far outstripping even the now increased riverine erosion flux.

\cite{WilkinsonMcElroy2007} do note however that such estimates may be too high: \cite{Beach1994} found that up to 87\% of historically eroded sediment in a Minnesota catchment remained within 4-25 km of the point it eroded from, and \cite{Trimble1999} found that only a small, usually steady fraction of eroded sediment was actually released downstream (in this case into the Mississippi river), even when both sediment sources and sinks greatly varied in size over decades. Sediment is not simply eroded in the uplands (naturally) or farmland (anthropogenically) and released to the ocean, there are numerous sinks that it can become stored in either temporarily or near permanently. Thus, sediment yield out of a system (e.g. out of the land into the sea) can relate poorly to the amount of sediment eroded.  \cite{Thompson2014} also suggests that river sediment loads may be decoupled from total discharge, with disease prevalence among corals - enhanced due to greater sediment and nutrient flux - only peaking during the transition from dry to wet conditions, or during floods.

%Make a cartoon diagram showing the factors that influence sediment run off, maybe a series over time.

Despite this likelihood that sediment flux estimates are lower than they may initially appear, human activity still caused a large increase in sediment delivery to rivers and thus the ocean. Using values from Hooke (2000) and data on population growth from the US Census Bureau, Wilkinson and McElroy (2007) estimate that cropland erosion has displaced some 20000 Gt of soil over the course of human history, enough to cover Earth's land surface to a depth of 6 cm. This is a huge amount of sediment released over a short geological timespan of only ~10000 years. Conversely, \cite{Syvitski2005} argues that although human activity has greatly increased soil erosion and sediment transport in rivers, the actual flux reaching the ocean is greatly reduced by 1.4 billion tons per year compared to pre-human-influence due to sediment becoming stuck behind dams and in reservoirs. 

While terrestrial erosion may not correlate well to sediment/nutrient flux to the ocean due to the sinks in rivers and lakes, it seems reasonable to assume that coastal weathering - where rock and soil is being directly weathered by the ocean - would. Beach erosion increases with global sea level rise, as higher water levels enable erosion when water can reach higher up the beach \citep{Zhang2004}. Relative sea level is often more rapid, and can be caused by human activity such as oil, gas or water extraction \citep{Williams2018}. 
In the short term, rising sea levels will cause a spike in terrestrial - specifically coastal - erosion, but if they remain high then terrestrial erosion rates will reduce simply because there is less land to erode. While there are non-ocean sinks for coastal erosion, namely re-deposition at another coastal site, \cite{Mentaschi2018} found that between 1984 and 2015, over twice as much coastal area had been eroded as deposited, and that anthropogenic effects are the main driver for this change.


\cite{Roff2013} examined a historical loss of staghorn coral on the GBR in response to agricultural development by European settlers in the late 1800s. 


\subsection{Modern Examples}

Hunter and Evans (1995) report that phytoplankton blooms around around a sewage outfall site in Kanehoe Bay, Hawaii.

Loya (2004) reports reduced visibility around floating fish farms in the Red Sea.

Coral reefs are susceptible to both oil spills and the methods we use to clean them up. \cite{Goodbody2013} reported on the aftermath of the Deepwater Horizon oil spill on coral colonies in the Gulf of Mexico, finding that the crude oil released caused the deaths of \textit{Porites Astreoides} larvae within 24 hours and also decreased larval settlement success and post-settlement survival. They also found that the dispersant used in the clean up of the DWH oil spill also reduced settlement success and survival after both shot nd long term exposure, with some tested species resulting in "complete larval mortality" after exposure to medium and high concentrations of the dispersant.

Other shallow marine organisms can have an adverse affect on corals. Sponges often compete for the same substrate spaces as corals, with species such as \textit{Clathria aceratoobtusa} found to aggressively compete with corals for space, overgrowing and killing them \citep{Ashok2020}. 
Invasive sponge species can rapidly overgrow and ruin coral communities. In Pearl Harbour, the orange keyhole sponge \textit{Mycale grandis} increased its coverage by 13\% in just one year while coral cover declined by 16\%, and attempts to remove such sponges can were time consuming and ultimately unsuccessful \citep{Coles2007}
%There were several store room specimens with an unusual crust like thing on them (specifically in the silicified samples), a couple of mm thick with regular patterns of holes. Could this have been sponge overgrowth?
 Sponges can also rapidly destroy reef substrate and coral skeletal interiors, with \cite{Acker1985} estimating that the sponge \textit{Cliona caribbaea} removed up to 45\% of the substrate at double the rate of reef calcification. Such Bioerosion makes coral colonies weaker to wave shock, with sponges attacking the base of a colony, and may impose an upper limit on the size of coral heads as a result \citep{Hein1975}.
 Negative interactions with \textit{Cliona}, in addition to White-band disease and reef bleaching led \cite{Williams1999} to predict that two coral ramparts in cays near Puerto Rico would be destroyed by the next set of storms, and would represent "a quick, obvious and permanent consequence of global disturbances." %It sure would be good to find something saying that they did get destroyed!

Given that sponges appear to be highly competitive, it might be expected that they would outcompete corals for space freed up after a disturbance. However, sponges are not the only organisms that can outcompete corals.  \cite{Gonzalez2016} notes that this has only been found to happen in a few studies. They found that while the space freed by coral death can promote outbreaks of boring sponges, they are quite often outcompeted themselves by macroalgae.

\cite{Ortiz2018} for a discussion on the cumulative stressors of the GBR.

\subsection{Extreme environments}
Despite worsening conditions, corals can adapt reasonably well. Corals in a high temperature, low pH and low oxygen environment (A semi enclosed lagoon in New Caledonia) still exhibited high richness and coverage despite the harsh conditions \citep{Camp2017}, though calcification rate and skeletal density were lowered, primarily by the lower light intensity and high acidity respectively. Corals acclimatised to these conditions through "heterotrophic plasticity", with some species gaining energy and from the increased sediment supply to the lagoon, which could also give them a resistance to acidification \citep{Ramajo2016}. 



\subsection{The Great Amazon Reef System.}

The Amazon and Orinoco rivers deliver enormous amounts of fresh water and sediment to the ocean, and so it was thought that there was a barrier where reefs could effectively not survive between the two due to the negative impacts of sedimentation on nutrients and light availability. The existence of a reef in relatively close proximity to the Amazon river delta was first proposed by \cite{collette1977}, who found associations between reef fish and reef forming sponges between 0 and 5\textdegree N, right at the mouth of the Amazon, between 40 and 80 m water depths. They proposed that the endemism of the tropical marine fauna caused by the riverine input was likely limited to species living shallower than 50 m. While this does mean effectively all autotrophic corals would be heavily affected by the "barrier effect" that the Amazon river has on its surroundings, heterotrophic corals and other reef building organisms can survive such impacts at depth. \cite{Moura2016} reported on "An extensive carbonate system off the Amazon mouth, underneath the river plume", finding that while cnidarians were present across the whole reef, sclerectinians with \textit{Symbiodinium} associates were mostly restricted to the central and southern sections, and were impoverished, low cover assemblages. almost all corals they found had large depth ranges.

%Moura2016 has a fair amount of information about the differences in water columns between the Amazon plume and non-plume waters.


\cite{Francini-Filho2018} explored the subsequently discovered Great Amazon Reef System (GARS) via video survey, giving an updated overview on the mesophotic reef system. They found that GARS may be as large as 56,000 km$^{2}$ and reach as far as 220 m deep (the deepest mesophotic reef that far discovered), and is potentially more complex and diverse than previously thought. While the reef is primarily built by calcareous algae, sclerectinian corals are also present. The deepest portions are dominated by sponges and octocorals, while the shallowest edges lack substrata and are dominated by fine sands or muds. Effectively, the Amazon river and sediment transport in the middle continental shelf determines the upper depth limit of GARS, preventing reef organisms from establishing themselves above 70 m depth. \cite{Francini-Filho2018} support the hypothesis that GARS may be a "mesophotic corridor" connecting Carribean and Brazilian reefs, noting the appearance of \textit{Chromis cyanea} that was previously only known from the Carribean, and that the description of "typical reef fish fauna" around the Amazon river from \cite{collette1977} also supports this. While such mesophotic corridors are likely unsuitable for typical autotrophic reef builders, it is possible that they could serve as refugia or repopulation pathways for reefs following a disturbance in shallow reefs. \cite{Moura2016} challenges this, noting that shallow water dwellers would likely not be able to survive such a multigenerational journey over ecological timescales, but posits that over longer periods of sea level fall, the eventual shallower waters of the Amazon reef may allow connectivity between the Caribbean and South Atlantic.

%^Diagram on how that would work?  

%Move above sentence on refugia to the section on the future?

That said, \cite{Francini-Filho2018} make the case that although light availability is dependent on the Amazon plume sediment load, it is not the limiting factor for calcareous algae in GARS. %"To what extent does sediment run off control development?"


%Fabricius 2011 cited in camp 2018 found that loss of architextural complexity compromises the structural integrity of reef framework. McCulloch et al 2012 and Wall et al. 2016 may define physiological limits of calcification

\section{The Future}

ADDED SEVERAL REFERENCES IN THE .BIB FILE. NEED TO READ THESE IMMEDIATELY.

\subsection{The changing climate}

The latest IPCC report \citep{IPCC2021WG1} (Specifically chapter 4) predicts that mean atmospheric temperatures will rise between 1.5\textdegree C and 4.8\textdegree C by the years  2081-2100 compared to 1850-1900, depending on the SSP used. IPCC2021 specifically notes that warming of 2\textdegree C above 1850-1900 levels will exceed hazard thresholds for many marine organisms and ecosystems, including coral reefs. Sea surface temperatures are predicted to rise between 0.86\textdegree C and 2.89\textdegree C depending on SSP (Chapter 9). IPCC2021 predicts that ocean warming observed since 1971 will at least double by 2100 under low warming scenarios, and increase by up to eight fold under high warming scenarios. Ocean acidity, stratification and deoxygenation will increase over the 21st century, as will marine heatwave frequency, already doubled between 1982 and 2016 (IPCC2021 Chapter 9). On a local scale, tropical cyclones will increase both in frequency and intensity with warming, which will lead to greater physical damage to coral reefs. 






\subsection{Coral Structure}

Due to the lag time of the ocean equilibrating with the atmosphere, sea surface temperature will rise over the next century and beyond \citep{IPCC2021WG1}. This rise in temperature, in conjunction with higher levels of atmospheric CO$_{2}$, means oceanic pCO$_{2}$ and acidity will also rise. \cite{Allison2022} investigated the effects of varying seawater pCO$_{2}$ on the skeletal morphology of $\textit{Porites}$ corals, specifically  using pCO$_{2}$ values for the Last Glacial Maximum (180 $\mu$atm), the present day (400 $\mu$atm), and a potential high CO$_{2}$ future (750 $\mu$atm). They found that median calyx size and proportion of surface occupied by calices decreased with an increase in seawater pCO$_{2}$ across all genotypes, but that calcification was only significantly reduced for 2 out of the 4 genotypes when seawater pCO$_{2}$ was raised to 750 $\mu$atm. Overall, stressed conditions (higher pCO$_{2}$) lead to thicker corallite walls, smaller calices and smaller polyps, and also lead to skeletons appearing more ornate with more abundant spines on the skeletal surface, though the surface becomes smoother with a less defined crystal structure at high seawater pCO$_{2}$.

These results oppose those of other studies that found that higher seawater pCO$_{2}$ led to larger calyx sizes or remained constant (Citing Tambutte et al., 2015 and Scucchia et al., 2021 respectively), with explanation given that less material is needed to build a skeleton with larger calices. \cite{Allison2022} suggests that there would be a higher energetic cost to increase the size of the polyp occupying the calyx, and that this may therefore be why calyx and polyp size reduced. Direct relationships between calcification rate and carbonate saturation or pCO$_{2}$ have been demonstrated for several different calcifying organisms \citep{Sciandra2003,Bijma2002}, including sclerectinian corals \citep{Reynaud2003}, and it is generally concluded that calcification rate decreases as carbonate saturation state decreases or are pCO$_{2}$ increases. \cite{Leclercq2000} similarly demonstrated that seawater pCO$_{2}$ had a control on coral calcification, predicting that by 2065 the calcification rate of coral communities could decrease by 21\% compared to pre-industrial (1880) levels. 

\subsection{Projections}
Models by \cite{Andersson2005} predict that decreases in ocean surface water carbonate saturation state could reduce biogenic carbonate production by 42\% by the year 2100 compared to 2000, and that it would be difficult for coral reefs as we know them to exist. They also predict that cool-water carbonate systems (Which already operate close to saturation), those at higher latitudes, would experience the effects of lowered carbonate saturation state earlier than tropical systems that exist in super saturated conditions. Some studies (CITE CITE CITE) predict that coral reefs may move to higher latitudes to avoid the effects of rising global temperatures, but such localities may by then have suffered under carbonate desaturation, making them insuitable for refuge (Though surely the rising temperatures would might make the saturation state acceptable again??). Finally, \cite{Andersson2005} notes that lowered carbonate saturation preferentially dissolves aragonite and Mg-calcite. This could mean that some corals might start to produce calcite, or may lose dominance as reef building organisms to those that do, similar to the Cretaceous \citep{Ries2006}.

Using data from previous studies, \cite{Langdon2005} produced two "clusters" of data predicting declines in coral reef calcification. The first predicted a 60\% (40-83\%) decrease in calcification by 2065 compared to an estimated $\Omega_{Arag}$ in 1880, however the second predicted a decrease of only 1-18\%. % I think it was maybe a bit wrong for Pandolfi to just grab this larger number.

Corals and the reefs they build have clearly been able to exist and thrive under seemingly much more stressful conditions than those that currently exist or will exist in the coming centuries. They did, after all, live and thrive in times where CO$_{2}$ levels were well in excess of 1000 ppm and when global temperatures were 5-10\textdegree C higher than today. \cite{Pandolfi2011} attributes this to the fact that on geological time scales, $\Omega_{Arag}$ (A mineral's saturation state, which determines how easily it will form) is uncoupled from ocean pH and high atmospheric CO${2}$ due to geochemical feedbacks that increase alkalinity. Rapid increases in CO${2}$ occur too fast for these feedbacks and therefore lead to declines in $\Omega_{Arag}$. %Reference 15 in this paper is apparently relevant to this but it's so compressed that I can't see the title.
\cite{Pandolfi2011} also found that tropical SSTs potential to warm over centennial-millennial time scales since the Last Glacial Maximum, with none of these episodes interrupting reef growth. They note however that SSTs were generally cooler than today, and that there were no rapid changes in pH or greenhouse gases equivalent to today. %Read references 19-20 of this paper because this sounds extraordinary.

\cite{Klein2022} project that ocean acidification was a relatively minor contributor to corals compared to Marine Heat Waves (MHWs), but that in scenarios of intermediate and unrestricted emissions (RCP4.5 and RCP8.5 respectively), ocean acidification would cause larger decreases in photosynthesis and survival.

Almost 90\% of coral reefs face long term degradation by 2100, even under a 1.5\textdegree C warming scenario \citep{Frieler2013}, leaving "little doubt" that coral reefs will stop being common coastal ecosystems if temperatures rise by 2\textdegree C above pre-industrial levels. \cite{Frieler2013} does note that thermal adaptation could lead to two-thirds of reefs avoiding long term degradation under 1.5\textdegree C warming, but is not optimistic given the time scale such adaptation would take and the other compounding anthropogenic factors. While heat-tolerant corals may survive, the reduced biodiversity would make them more susceptible to impacts such as disease, and ecosystems may become locked into a new stable state dominated by macroalgae \citep{Mumby2007}

% with this slight difference in average temperature marking "the difference between events at the upper limit of present-day natural variability and a new climate regime" \citep{Schleussner2016}.



\subsection{Bleaching}

As noted in section \ref{Present Bleaching}, bleaching events have become more common since 1980, and \cite{Hughes2018a} predicts that the gap between bleaching events will only become shorter in the future, with the potential for every hot summer to cause a bleaching event regardless of ENSO phase. Coral assemblages already struggle to recover on such short time scales, and regular bleaching events will inevitably lead to higher coral mortalities. Prospects of a full recovery from prolonged bleaching events are poor. Corals weakened by bleaching are more susceptible to disease, and even the fastest growing corals would take at least a decade to fully replace dead colonies \citep{Hughes2018b}. %I could cite Muller 2008, Kayanne 2002, and others that Hughes cites, but is it worth it?
Bleaching events are already significantly more common than such an optimistic recovery time, meaning that it is likely that such assemblages will not be able to recover.


While coral bleaching is currently almost universally bad, there are some suggestions that bleaching was "originally intended" as an adaptive mechanism, a basic attribute of organisms with zooxanthellae (Corals, some clams, sea anemones) where they can expel their symbionts in response to  stress (usually heat stress) so that they can be repopulated by more resistant symbionts \citep{Buddemeier1993}. However, while zooxanthellae more resistant to heat stress such as some strains \textit{Symbiondinium} do exist \citep{Sotka2005}, the evidence that bleaching can be an adaptive mechanism \citep{Baker2001} is debated \citep{Coles2003}. Coral reefs in the Maldives underwent 90\% mortality in the 1998 bleaching event, and took 16 years to recover their hard cover coral \citep{Montefalcone2020}. Mortality in the 2016 event was lower, however the thermal anomaly in the 1998 event was higher for the Maldives.

\cite{Pandolfi2011} suggests that the assumption that corals susceptible to bleaching will decline in abundance relative to more resistant corals is not necessarily true, pointing to the facts that such susceptible species often have faster recovery rates, and their shorter generation times may allow bleaching thresholds to evolve faster. While it may seem unlikely that raising bleaching thresholds through evolution could keep up with rising global SSTs, \cite{Pandolfi2011} cites \cite{Stockwell2003} to suggest that "substantial evolutionary change can occur over the decadal time scales relevant to reef managers", though \cite{Stockwell2003} does not mention corals or reefs. (This feels quite weird to have in here now? Maybe this should be cut out).







\subsection{IDK but it is a different section}

%\cite{Marubini1996} found that photosynthetic pigment concentration increased with eutrophication, even when gross photosynthesis was lower due to turbidity indicued light reduction. They suggested that this pigment concentration could therefore be the most linear and useful early warning indicator for eutrophication. 




\cite{Camp2018} reviews examples of coral reefs in extreme environments today to produce "the first collective assessment on the range of extreme conditions under which corals currently persist", with the aim to predict the future of coral reefs in currently good conditions under expected rapid climate change. While globally available data was too lacking for many of the parameters they reviewed, they were able to make comparisons across environments for temperature and pH. By reviewing the three known coral reefs adjacent to CO$_{2}$ volcanic vents, they found that only a few species could tolerate the acidification at such high pCO$_{2}$ levels, and that one site transitioned to macroalgal dominance with a pCO$_{2}$ increase, while another shifted from hard corals to soft corals. Examining the work done on the low pH environments of underwater seeps in Mexico, they report that only 3 species of corals exist in close proximity to sites that have had low pH water for millennia. 

\cite{Camp2018} notes that the upper thermal limit for corals is already close to the tropical environments that are considered optimum for them, and that as a result they are highly susceptible to ocean warming. Coral bleaching events have become significantly more common since the millenium %FIND A SOURCE, ASK ALEX WHAT THE NUMBERS WERE AGAIN. Hughes et al. 2017 would be a good place to start. It's cited in Camp2018
\cite{Camp2018} reviewed studies on corals in the Persian-Arabian Gulf where over 50 species of corals were recorded despite living under the most extreme recorded conditions for coral reefs, with sustained 35\textdegree C temperatures, a seasonal range of 20\textdegree C and high salinity. Not only this, but corals in this region have adapted remarkably fast given that such temperature conditions have only existed for 3-6000 years.

Reefs are also more directly threatened by human activity. The GARS described by \cite{Francini-Filho2018} is under threat due to proposed oil exploration from major energy companies, and they note that current velocities in the GARS are high enough that a potential oil spill could spread rapidly.

The overall picture appears to be that corals will most likely survive whatever human driven change puts them through, but that they will suffer greatly through it.

\subsection{Survival and Refugia}

%NEED TO FIND CITATIONS FOR FOLLOWING 2 PARAGRAPHS

Other marine organisms can have a part to play in the survival of coral reefs. \cite{Manzello2012} suggests that corals and other calcareous organisms could find refuge from ocean acidification among sea grasses, which have the ability to raise local mean pH due to being CO$_{2}$ sinks. \cite{Camp2018} notes, however, that at night the high respiration of sea grasses would lead to them lowering the pH of their environment significantly. Seagrass meadows also lack much of a solid substrate for coral recruitment and growth, and stimulate carbonate dissolution at their roots. 
\cite{Manzello2012} therefore suggest that hard substrate areas downstream from seagrass beds may act as refugia, rather than the beds themselves. \cite{Lohr2017} found that 14 coral taxa were able to survive in seagrass meadows, but that coral cover and diversity were lower than at back reef sites, and colony sizes were generally small.
Camp also notes that massive coral species are generally better adapted to light and heat stress, while \cite{Loya2001} found that reef communities shifted from fragile branch coral communities to massive and encrusting corals following the 1997-98 bleaching event. 

Corals in the Persian/Arabian Gulf have clearly specialised for heat stress, coping with annual fluctuations of 20\textdegree C and seasonal maxima up to 36\textdegree C. \cite{Hume2015} describes a new species of algal symbiont \textit{Symbiodinium thermophilum} that is predominant across the southern Gulf and notes that therefore the gulf may be a "genetic resource" that could facilitate increases in thermal tolerance among corals world over, though they do not conclude whether the symbiont evolved within the gulf or was exported to it and therefore already exists in the wider ocean. \cite{Krueger2017} studied the coral \textit{Stylophora pistillata} from the Gulf of Aqaba in the Red Sea, which was expected to be sensitive to environmental disturbances, and found that they exhibited no bleaching even though they experienced temperatures 1-2\textdegree C above summer maximum and a lowered pH in line with the worst case projections from the IPCC. In fact there was a 51\% increase in primary productivity and net oxygen production, and calcification was not significantly reduced. \textit{S. pistillata} in the north of the Red Sea are even currently living at below-optimum temperatures. \textit{S. Pistilla} is not the only Red Sea coral to survive or even benefit from higher temperatures, \cite{Fine2013} found that several other species found in the Gulf of Aqaba did not bleach even when exposed to temperatures 7\textdegree C above summer maxima (though \textit{symbiodinium} density did decrease). \cite{Fine2013} argue that corals in the Red Sea underwent evolutionary selection for thermal tolerance 6-7000 years ago when the only path to recolonising the Red Sea was through extremely warm waters, and that the Gulf of Aqaba may now serve as a refuge for coral reefs, as it is not expected to bleach under the expected temperature rise of the next 100 years. 


Evolutionary change is not the only way for corals to adapt. Corals have been shown to acclimatise to higher temperatures in as little as two years \citep{Palumbi2014}, gaining tolerance equivalent to several generations of strong natural selection.  Corals native to pools with high temperature variability (HV pool) had higher heat tolerance than those from pools with medium temperature variability (MV pool), but that heat tolerance was gained or lost relatively quickly among specimens that were transplanted from HV to MV pools and vice versa. \cite{Palumbi2014} notes that there was little change in the symbionts of the corals while they acclimated, and that the acclimatization is due to changes in gene expression of the corals. 


 Corals that live in higher latitudes are "functionally different to tropical corals" and that traits they have may be beneficial to surviving or adapting to climate change \citep{Camp2018}. High latitude reefs often have high diversity due to the overlap of ranges of temperate and tropical species. %Citing Beger et al. 2014
Higher latitude reefs and the species they contain are thought to be more tolerant to environmental change as there are greater fluctuations in these environments already %CITING Oliver and Palumbi 2011b, read.
 and this may increase their ability to cope with climate change. However, given the general poleward light attenuation and greater predicted warming compared to the equator, it's quite possible that coral communities at higher latitudes could be more susceptible in the future %citing Beger et al. 2014

Mesophotic reefs have also been considered as a potential refuge for coral reefs, as they face fewer disturbances than shallow reefs and can provide a stock of corals that can recolonise shallow environments after a disturbance. \cite{Camp2018} notes that several studies show that deeper reefs suffer reduced bleaching %citing Riegl and Piller 2003, Bridge et al. 2014
but that during a large bleaching event in 1997-98 in the Seychelles, deeper water corals underwent worse bleaching and it occured before the bleaching of shallower corals. \cite{Baird2018} examined the response of 16 coral taxa to the 2016 bleaching event on the Great Barrier Reef, and found that in 10 taxa, bleaching reduced with increasing depth. Corals may therefore be able to survive bleaching events by migrating to deeper waters. 

This "Deep Refuge Hypothesis" relies on assuming that deep reefs are less affected by anthropogenic activity than shallow reefs, and that there is enough of an overlap between the taxa of shallow and deep reefs so "shallow" taxa can take refuge there and not be outcompeted. \cite{Rocha2018} disagrees with the hypothesis, finding that most species (both coral and reef fish) showed a specific depth preference and were therefore not depth generalists even if the occasional "shallow" species could be found at mesophotic depths. They also found that deeper reefs were not particularly safer from natural disasters: All reefs well outside the path of Hurricane Matthew in 2016 had no signs of destruction, while mesophotic reefs in the hurricane path down to 135 m still showed physical damage and were covered in sediment. The mesophotic reefs were also all susceptible to heavy fishing, bleaching and invasive species, and \cite{Rocha2018} notes "the real refuges seem to be located in regions far from humans, regardless of depth".

\cite{Montefalcone2020} investigated the potential for the DRH on Maldivian reefs, finding that coral bleaching was negligible in the upper mesophotic zone (30-50 m). However, coral cover was under 15\% at this depth, lower than that of the surviving corals in the shallow reef, so they concluded that the potential for mesophotic corals providing a refuge to reseed from was lower than the reseeding from survivors of the bleaching event. Conversely, \cite{Muir2018} found that all shallow-reef-building coral families and 45\% of species on on the northern GBR extended below 30 m into the mesophotic zone, while 78\% of families and 13\% of species were still present below 45 m, suggesting that mesophotic reefs are still species rich and have potential to preserve coral lineages. With all shallow-reef-building families still existing at 30 m depth, Muir et al. support a "optimum refuge zone", where storm damage and bleaching are reduced but light availability and family/species diversity is not too limited.


%%Again, all this is nice but it all comes from one review. How exactly do I condense down so much relevant information?!

%Find some info about how long it would take oceans acidification and temperatures to recede from current values, what action would have to be taken to do that, etc.
\subsection{Replacement}

Reefs and other biogenic carbonate build ups have existed on Earth since the Precambrian, and the geological record makes it clear that corals were not always the dominant reef builders. There is no reason therefore that corals will continue to be the dominant reef builders in even the near future, particularly with how they have suffered from anthropogenic influences. 
Given their proclivity to invade reefs, compete with corals for substrate and overgrow and kill corals, some researchers suggest that sponge may become the dominant reef building organism in the future \citep{Bell2013}. %WRITE MORE ON THIS AND READ THE PAPER, IT LOOKS INTERESTING

Coral reefs are stable ecosystems, able to adapt to minor change or recover back to a steady state from various "issues" (WHAT IS THE WORD, IT'S OUT OF MY MIND). However, the niches that coral reefs occupy could be shifted to other stable states, particularly those dominated by macroalgae, following major changes in conditions \cite{Knowlton1992}. Coral cover declined by 80\% in the Caribbean between the 1970s to 2003 \citep{Gardner2003}. Reefs declined  in 1980 due to hurricane Allen and white band disease, but were recovering until 1983 due to large increases in macroalgae (in turn due to a mass die-off of the main macroalgae grazer, sea urchin \textit{Diadema antillarum}). Modelling by \cite{Mumby2007} suggested that these Caribbean reefs were not in an algal-dominated state while the sea urchin grazers were present, but that two stable states emerged after the die off event: A stable equilibrium with high coral cover maintained by grazing organisms and another dominated by macroalgae with low coral cover. 

%Maybe reproduce some of the modelling plots they did?
However, this replacement of of corals by macroalgae is likely less common than assumed. An global analysis of over 3000 studies coviring nearly 2000 reefs by \cite{Bruno2009} found that while macroalgal cover had increased, only 4\% of reefs were dominated by macroalgae (over 50\% cover), and only 25 out of 1851 reefs had undergone a "complete" coral to algae phase shift, though all except of one these was in the Caribbean. They also found that few reefs around the world fell into "stable points" as predicted by \cite{Mumby2007}.

\section{Conclusion}
 Lorum ipsum dolor sit amet
  Lorum ipsum dolor sit amet
   Lorum ipsum dolor sit amet
    Lorum ipsum dolor sit amet\\
     Lorum ipsum dolor sit amet\\
      Lorum ipsum dolor sit amet\\
       Lorum ipsum dolor sit amet\\
        Lorum ipsum dolor sit amet\\
         Lorum ipsum dolor sit amet
\bibliography{what_kills_corals_}
\bibliographystyle{apalike}

\end{document}

